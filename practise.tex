% --------------------------------------------------------------
% This is all preamble
% --------------------------------------------------------------
 
\documentclass[10pt]{article}


% Basic Packages for Encoding (Input AND Output) and Langauge Support
\usepackage[utf8]{inputenc}
\usepackage[T1]{fontenc}
\usepackage[french]{babel}

% Change Layout with a User-Friendly Interface
\usepackage[margin=1in]{geometry} 

% Include Pictures with a User-Friendly Interface
\usepackage{graphicx}
\graphicspath{ {./images/statistics_and_probability/} }
\usepackage{float}

% Extended Math Support from the Famous 'American Mathematical Society'
\usepackage{amsmath,amsthm,amssymb}

% Section title formatting
\usepackage{titlesec}

\titleformat{\section}
            {\normalfont\scshape}{\thesection}{1em}{}

\newcommand{\N}{\mathbb{N}}
\newcommand{\Z}{\mathbb{Z}}

%Numbered Questions 
\newcounter{question}
\newenvironment{question}
               {\refstepcounter{question}\par\medskip\noindent\textbf{Q~\thequestion.}\par \noindent \rmfamily}
               {\medskip}

\usepackage{pgfplots}
 
\begin{document}
 
% --------------------------------------------------------------
%                        Actual content
% --------------------------------------------------------------

\title{Exercices de révision: Calcul}
\author{Pierre B. \thanks{this is just a scratch .tex file I use for experimenting}}
\date{\today}
\maketitle
%
% Ellipse Begin
%
\begin{tikzpicture}
  \def\Xmax{3}    \def\Ymax{3}
  \def\X{1}       \def\Y{2}
  \begin{axis}[
      width=8cm,  height=8cm,
      scale only axis=true,
      %
      xmin=-\Xmax,   xmax=\Xmax,
      ymin=-\Ymax,   ymax=\Ymax,
      %
      extra x ticks={-\X,\X},
      extra y ticks={-\Y,\Y},
      extra tick style={grid=major},
    ]
    % Trying to plot an ellipse centered at coordinates (2,-1) and having
    % half axes of legnth 1 and 2.
    \pgfplotsextra{% <--- important!
      %
      % we compute the (x,y) coordinates as difference to be
      % independent of the (0pt,0pt) used internally by pgfplots
      \pgfpointdiff
	  {\pgfplotspointaxisxy{0}{0}}
	  {\pgfplotspointaxisxy{\X}{\Y}}
	  \extractpgfxy{\RADIUSX}{\RADIUSY}%
	  %
	  % I used the option form for ellipse because the '\RADIUSX and' seems
          % to be a problem in pgf (?)
	  \draw[black] (axis cs:0,0)
          ellipse [          x radius=\RADIUSX,y radius=\RADIUSY];
	  \draw[red]   (axis cs:0,0)
          ellipse [rotate=90,x radius=\RADIUSX,y radius=\RADIUSY];
    }
    \addplot [only marks,mark=*] coordinates { (0,0) };
  \end{axis}
\end{tikzpicture}

%
% Ellipse End
%
\end{document}

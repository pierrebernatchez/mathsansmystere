\documentclass{article}


% Basic Packages for Encoding (Input AND Output) and Langauge Support
\usepackage[utf8]{inputenc}
\usepackage[T1]{fontenc}
\usepackage[english]{babel}

% Change Layout with a User-Friendly Interface
\usepackage{geometry}

% Include Pictures with a User-Friendly Interface
\usepackage{graphicx}
\usepackage{float}

% Extended Math Support from the Famous 'American Mathematical Society'
\usepackage{amsmath}
\usepackage{amssymb}




\title{practisemath}
\author{pbz }
\date{March 2022}

\begin{document}

\maketitle

\section{Formula Samples}

Notes

The \$ is for inline math markup and the \$\$ is for display math markup.

Inline is good for expressions you want to and can  embed well within text.

Display is centered on a line by itself, good when the expression does not
mesh well within a line of text.

This is inline math $e^{i\pi} + 1 =0$

This is display math $$ e = \lim_{n\to\infty} \left(1+\frac{1}{n}\right)^n
= \lim_{n\to\infty} \frac{n}{\sqrt[n]{n!}}$$

$$e=\sum_{n=0}^{\infty} \frac{1}{n!}.$$

\section{element of}

    $$ \forall x \in ] -\infty, \infty [ $$
    $$ \forall x \in \mathbb{R} $$
    
    
\section{try stuff here}
    
State the value of x in the form $a \times 10^{k}$  where $1 \leq a < 10$ and  $k \in \mathbb{Z}$    
    

\end{document}

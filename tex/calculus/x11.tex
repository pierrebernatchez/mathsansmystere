% calculus:x11
\begin{question}
  \hspace*{\fill} [Note maximale: 14]\par
  \medskip
  \noindent On considère la fonction $f$ dont la dérivée seconde est $f^{\prime\prime}(x) = 3x -1$. La représentation graphique de $f$ présente un point minimum en $A(2 ; 4)$ et un point maximum en $B(-\frac{4}{3}; \frac{358}{27})$.\par
  \medskip
  (a) Utilisez la dérivée seconde pour justifier que $B$ est un maximum.\hspace*{\fill} [3]\par
  \medskip
  (b) Étant donné que $f^\prime(x) = \frac{3}{2}x^2 - x + p$, montrez que $p = -4$.\hspace*{\fill} [4]\par
  \medskip
  (c) Trouvez $f(x)$.\hspace*{\fill} [7]\par
\end{question}


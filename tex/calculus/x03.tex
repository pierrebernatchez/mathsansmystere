% calculus:x03
\begin{question}
  \hspace*{\fill} [Note maximale: 16]\par

  \noindent Considérez une fonction $f(x)$. la droite $L_1$ d’équation $y = 3x + 1$ est une tangente à la représentation graphique de $f(x)$ lorsque $x = 2$.\par
  \medskip
  (a)\par
  \hspace{2em} (i)  Écrivez $f^\prime(2)$.\par
  \medskip
  \hspace{2em} (ii) Trouvez $f(2)$.\hspace*{\fill} [4]\par
  \medskip
  \noindent Soit $g(x) = f(x^2 + 1)$ et $P$ le point sur la représentation graphique de $g$ où $x = 1$\par
  \medskip
  (b) Montrez que la représentation graphique de $g$ a une pente de 6 au point $P$.\hspace*{\fill} [5]\par
  \medskip
  (c) Soit $L_2$ la tangente à la représentation graphique de $g$ au point $P$. $L_1$ coupe $L_2$ au point $Q$.\par
  \hspace{1em} Trouvez l’ordonnée de $Q$.\hspace*{\fill} [7]\par
\end{question}


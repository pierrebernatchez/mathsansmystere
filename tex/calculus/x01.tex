% calculus:x01
\begin{question}
  \hspace*{\fill} [Note maximale: 6]\par
  \noindent Soit la fonction $f(x) = 6x^2-3x$ representé ci-dessous\par
  \medskip

  %
  % Plot begins
  %
  \scalebox{.5}[1.0]{ % {horizontal scale} [vertical scale]
  \begin{tikzpicture}
  \begin{axis}[
    xlabel=$x$,
    ylabel=$f(x)$
    ] 
    \addplot [xscale=1.0, yscale=1.0, smooth] coordinates{(0, 0) (0,20)}; % y axis
    \addplot [xscale=1.0, yscale=1.0, domain=0:2.2, smooth] { 0 }; % x axis
    \addplot [xscale=1.0, yscale=1.0, smooth] coordinates{ (1,0) (1, 3) }; 
    \addplot [xscale=1.0, yscale=1.0, smooth] coordinates{ (2,0) (2, 18) };
    \addplot [xscale=1.0, yscale=1.0, domain=0:2.2, smooth] { 6 * x^2 - 3*x }; % f(x)
  \end{axis}
  \end{tikzpicture}}\par
  %
  % Plot ends
  %
  \medskip
  (a) Trouvez $\int \! (6x^2-3x) \, \mathrm{d}x$\hspace*{\fill} [2]\par
  \medskip

  (b) Trouvez l’aire de la région délimitée par la représentation graphique de $f(x)$,\par
  \hspace{1em} l’axe des abscisses et les droites x = 1 et x = 2.\par
  \hspace{1em} C'est à dire $\int_1^2 \! (6x^2-3x) \, \mathrm{d}x$ \hspace*{\fill} [4]

\end{question}

